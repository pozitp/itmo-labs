\usepackage{wasysym}%%%%%%%%%%%%%%%%%%%%%%%%%%%%%%%%% LAB-5 %%%%%%%%%%%%%%%%%%%%%%%%%%%%%%%%%%
%>>>>>>>>>>>>>>>>>>>>>>>>>> ПЕРЕМЕННЫЕ >>>>>>>>>>>>>>>>>>>>>>>>>>>>>>>>>>>
%>>>>> Информация о кафедре
%\newcommand{\year}{2021 г.}  % Год устанавливается автоматически
\newcommand{\city}{Санкт-Петербург}  %  Футер, нижний колонтитул на титульном листе
\newcommand{\university}{Национальный исследовательский университет ИТМО}  % первая строка
\newcommand{\department}{Факультет программной инженерии и компьютерной техники}  % Вторая строка
\newcommand{\major}{Направление программная инженерия}  % Треьтя строка
% Пусть будет. Проще закоментить лишнее.
\newcommand{\education}{Образовательная программа системное и прикладное программное обеспечение}  % четвертая строка
%\newcommand{\specialization}{}  % пятая строка

%<<<<< Информация о кафедре

%>>>>> Назание работы
\newcommand{\reporttype}{ОТЧЕТ ПО ЛАБОРАТОРНОЙ РАБОТЕ} % тип работы, (главный заголовок титульного листа)
\newcommand{\lab}{Лабораторная работа}          % вид работы
\newcommand{\labnumber}{№ 1}                    % порядковый номер работы
\newcommand{\subject}{Программирование}         % учебный предмет
\newcommand{\labtheme}{Принципы ООП}            % Тема лабораторной работы
\newcommand{\variant}{№ 108699}                % номер варианта работы

\newcommand{\student}{Мухамедьяров Артур Альбертович}    % определение ФИО студента
\newcommand{\studygroup}{P3109}                 % определение учебной группы
\newcommand{\teacher}{% принимающий
    Гаврилов А. В.,\\[1mm]% ФИО лектора
    Мустафаева А. В.% ФИО практика
}
%<<<<<<<<<<<<<<<<<<<<<<<<<< ПЕРЕМЕННЫЕ <<<<<<<<<<<<<<<<<<<<<<<<<<<<<<<<<<<


%>>>>>>>>>>>>>>>>>>>>>> ПРЕАМБУЛА >>>>>>>>>>>>>>>>>>>>>>>>>
\include{preamble}
%<<<<<<<<<<<<<<<<<<<<<< ПРЕАМБУЛА <<<<<<<<<<<<<<<<<<<<<<<<<


%%%%%%%%%%%%%%%%%%% СОДЕРЖИМОЕ ОТЧЕТА %%%%%%%%%%%%%%%%%%%%%
%>>>>>>>>>>>>>>> ''''''''''''''''''''''' >>>>>>>>>>>>>>>>>>
\begin{document}


%>>>>>>>>>>>>>>>> ОПРЕДЕЛЕНИЕ НАЗВАНИЙ >>>>>>>>>>>>>>>>>>>>
% Переоформление некоторых стандартных названий
%\renewcommand{\chaptername}{Лабораторная работа}
    \renewcommand{\chaptername}{\lab\ \labnumber} % переименование глав
    \renewcommand{\contentsname}{Содержание} % переименование оглавления
%<<<<<<<<<<<<<<<< ОПРЕДЕЛЕНИЕ НАЗВАНИЙ <<<<<<<<<<<<<<<<<<<<
% \setlength{\itemsep}{0pt} % установка расстояния между строчками в списках можно использовать локально внутри списка списке
% \setlength{\parskip}{0pt} % 
% \setlength{\parsep}{0pt}  % 

%>>>>>>>>>>>>>>>>> ТИТУЛЬНАЯ СТРАНИЦА >>>>>>>>>>>>>>>>>>>>>
    \include{titlepage}
%<<<<<<<<<<<<<<<<< ТИТУЛЬНАЯ СТРАНИЦА <<<<<<<<<<<<<<<<<<<<<


%>>>>>>>>>>>>>>>>>>>>> СОДЕРЖАНИЕ >>>>>>>>>>>>>>>>>>>>>>>>>
% Содержание
    \tableofcontents
%<<<<<<<<<<<<<<<<<<<<< СОДЕРЖАНИЕ <<<<<<<<<<<<<<<<<<<<<<<<<


%%%%%%%%%%%%%%%%%%%%%%% КОД РАБОТЫ %%%%%%%%%%%%%%%%%%%%%%%%
%>>>>>>>>>>>>>>>>>>>'''''''''''''''''>>>>>>>>>>>>>>>>>>>>>
    \newpage
    \Chapter{\lab\ \labnumber}{}{}

    \Section{Задание варианта \variant}

    \begin{center}
        , , ,
    \end{center}
    \noindent
%    \textbf{
%    % Заглавное описание....:
%        Заголовок
%    }
%
%    \textit{
%    % Описание задания...
%        Описание
%    }

    \begin{enumerate}
        \setlength{\itemsep}{0pt} % Сокращение межстрочных расстояний
        \setlength{\parskip}{0pt}
        \setlength{\parsep}{0pt}
        \item Создать одномерный массив s типа short. Заполнить его числами от 1 до 16 включительно в порядке убывания.
        \item Создать одномерный массив x типа double. Заполнить его 11-ю случайными числами в диапазоне от -14.0 до 11.0.
        \item Создать двумерный массив w размером 16x11. Вычислить его элементы по следующей формуле (где $x = x[j]$):

        \begin{itemize}
            \setlength{\itemsep}{0pt} % Сокращение межстрочных расстояний
            \setlength{\parskip}{0pt}
            \setlength{\parsep}{0pt}
            \item если $\displaystyle s[i] = 15\text{,}$\\$\displaystyle \text{то }w[i][j] = \ln{\left(\sqrt{\left(\frac{|x|+1}{2}\right)^{2}}\right)}$;
            \item если $\displaystyle s[i] \in\{1, 3, 7, 8, 11, 12, 13, 16\}\text{,}$\\ $\displaystyle \text{то } w[i][j] = \left(\frac{3}{4}\cdot\left(\left(\frac{\sqrt[3]{x}}{\frac{1}{3}+(x)^{x+\pi}}\right)^{3}+1\right)\right)^{3}$;
            \item  для остальных значений $\displaystyle s[i]:$\\ $\displaystyle w[i][j] = \tan{\left(\left(\tan{\left(\frac{1}{2}/x\right)}\right)^{\left(\frac{2}{3}\cdot\tan{(x)}\right)^{2}}\right)}$.
        \end{itemize}

        \item Напечатать полученный в результате массив в формате с четырьмя знаками после запятой.
    \end{enumerate}


    \begin{center}
        ' ' '
    \end{center}

    \newpage
    \Section{Выполнение задания.}
    Задание было выполнено в редакторе кода, позже собрано с помощью \verb|javac| в \verb|jar| файл \verb|itmo_proga_lab1.jar| непосредственно на сервере.
%    \newpage
    \Subsection{Листинги кода}
    Листинг из файла\ref{lst:java}
    \lstinputlisting[caption={Исходный код программы},label={lst:java},language=Java]{../src/Main.java}

%    Листинг в код latex \ref{lst:sql}
%    \begin{lstlisting}[caption={SQL},label={lst:sql}]
%declare @t table(
%  id int
%)
%    \end{lstlisting}

%    Листинг прямо в текст: \lstinline[columns=fixed]{declare}. Либо еще так: \verb|declare|.

%    Вставленное изображение с описанием и шириной по тексту.
%    \begin{figure}[H] % 'H' -- вставить тут же (подключен модуль), обычный вариант: 'htpb'
%        \centering
%        % { граница для иллюстрации
%        % \setlength{\fboxsep}{0pt}% убрать отсутп от границы
%        % \setlength{\fboxrule}{1pt}%
%        % \fbox{%
%        \includegraphics[width=\textwidth]{res/UML-class-diagram.png}
%        % }} % ограничение области действия параметров
%        \caption{Caption}
%        \label{fig:enter-label}
%    \end{figure}


% Выполнение задания...
    \Section{Результат работы программы.}
    \Subsection{Первый запуск.}


    \begin{small}
        ~
        \[\begin{array}{llllllllllllllll}
              0.4970 & 0.4219 & 0.4219 & 0.4219 & NaN    & NaN    & NaN    & NaN    & NaN    & 0.4219 & NaN    \\
              0.1761 & 0.5441 & 0.6532 & 0.7404 & 0.7782 & 0.8451 & 0.3979 & 0.8129 & 0.6021 & 0.4771 & 0.5441 \\
              0.0553 & 1.2885 & 0.0000 & 0.6430 & NaN    & NaN    & NaN    & NaN    & NaN    & 0.0000 & NaN    \\
              0.4970 & 0.4219 & 0.4219 & 0.4219 & NaN    & NaN    & NaN    & NaN    & NaN    & 0.4219 & NaN    \\
              0.4970 & 0.4219 & 0.4219 & 0.4219 & NaN    & NaN    & NaN    & NaN    & NaN    & 0.4219 & NaN    \\
              0.4970 & 0.4219 & 0.4219 & 0.4219 & NaN    & NaN    & NaN    & NaN    & NaN    & 0.4219 & NaN    \\
              0.0553 & 1.2885 & 0.0000 & 0.6430 & NaN    & NaN    & NaN    & NaN    & NaN    & 0.0000 & NaN    \\
              0.0553 & 1.2885 & 0.0000 & 0.6430 & NaN    & NaN    & NaN    & NaN    & NaN    & 0.0000 & NaN    \\
              0.4970 & 0.4219 & 0.4219 & 0.4219 & NaN    & NaN    & NaN    & NaN    & NaN    & 0.4219 & NaN    \\
              0.4970 & 0.4219 & 0.4219 & 0.4219 & NaN    & NaN    & NaN    & NaN    & NaN    & 0.4219 & NaN    \\
              0.0553 & 1.2885 & 0.0000 & 0.6430 & NaN    & NaN    & NaN    & NaN    & NaN    & 0.0000 & NaN    \\
              0.0553 & 1.2885 & 0.0000 & 0.6430 & NaN    & NaN    & NaN    & NaN    & NaN    & 0.0000 & NaN    \\
              0.0553 & 1.2885 & 0.0000 & 0.6430 & NaN    & NaN    & NaN    & NaN    & NaN    & 0.0000 & NaN    \\
              0.4970 & 0.4219 & 0.4219 & 0.4219 & NaN    & NaN    & NaN    & NaN    & NaN    & 0.4219 & NaN    \\
              0.0553 & 1.2885 & 0.0000 & 0.6430 & NaN    & NaN    & NaN    & NaN    & NaN    & 0.0000 & NaN    \\
              0.4970 & 0.4219 & 0.4219 & 0.4219 & NaN    & NaN    & NaN    & NaN    & NaN    & 0.4219 & NaN
        \end{array}\]
    \end{small}


    \Subsection{Второй запуск.}

    \begin{small}
        ~
        \[ \begin{array}{llllllllllllllll}
               2.2610 & NaN    & 0.4263 & NaN    & 0.4219 & 0.4219 & NaN    & 2.2610 & NaN    & 0.4219 & 0.4219  \\
               0.0000 & 0.0000 & 0.3010 & 0.0000 & 0.7404 & 0.5441 & 0.8751 & 0.0000 & 0.3010 & 0.5441 & -0.3010 \\
               0.5741 & NaN    & 1.5040 & NaN    & 0.6430 & 1.2885 & NaN    & 0.5741 & NaN    & 1.2885 & 1.5574  \\
               2.2610 & NaN    & 0.4263 & NaN    & 0.4219 & 0.4219 & NaN    & 2.2610 & NaN    & 0.4219 & 0.4219  \\
               2.2610 & NaN    & 0.4263 & NaN    & 0.4219 & 0.4219 & NaN    & 2.2610 & NaN    & 0.4219 & 0.4219  \\
               2.2610 & NaN    & 0.4263 & NaN    & 0.4219 & 0.4219 & NaN    & 2.2610 & NaN    & 0.4219 & 0.4219  \\
               0.5741 & NaN    & 1.5040 & NaN    & 0.6430 & 1.2885 & NaN    & 0.5741 & NaN    & 1.2885 & 1.5574  \\
               0.5741 & NaN    & 1.5040 & NaN    & 0.6430 & 1.2885 & NaN    & 0.5741 & NaN    & 1.2885 & 1.5574  \\
               2.2610 & NaN    & 0.4263 & NaN    & 0.4219 & 0.4219 & NaN    & 2.2610 & NaN    & 0.4219 & 0.4219  \\
               2.2610 & NaN    & 0.4263 & NaN    & 0.4219 & 0.4219 & NaN    & 2.2610 & NaN    & 0.4219 & 0.4219  \\
               0.5741 & NaN    & 1.5040 & NaN    & 0.6430 & 1.2885 & NaN    & 0.5741 & NaN    & 1.2885 & 1.5574  \\
               0.5741 & NaN    & 1.5040 & NaN    & 0.6430 & 1.2885 & NaN    & 0.5741 & NaN    & 1.2885 & 1.5574  \\
               0.5741 & NaN    & 1.5040 & NaN    & 0.6430 & 1.2885 & NaN    & 0.5741 & NaN    & 1.2885 & 1.5574  \\
               2.2610 & NaN    & 0.4263 & NaN    & 0.4219 & 0.4219 & NaN    & 2.2610 & NaN    & 0.4219 & 0.4219  \\
               0.5741 & NaN    & 1.5040 & NaN    & 0.6430 & 1.2885 & NaN    & 0.5741 & NaN    & 1.2885 & 1.5574  \\
               2.2610 & NaN    & 0.4263 & NaN    & 0.4219 & 0.4219 & NaN    & 2.2610 & NaN    & 0.4219 & 0.4219
        \end{array}\]
    \end{small}
    \newpage

    \Section{Вывод}
% Вывод...
    Во время выполнения лабораторной работы я изучил синтаксис языка \verb|Java|, встроенную библиотеку \verb|Math|, научислся работать со средством разработки Java (\verb|JDK|). Также в процессе выполения я научился рабоать с типами данных, классами, функциями, массивами и циклами. Полученные мною знания являются необходимой базой для дальнейшего изучения языка и разработки уже более комлпексных проектов.\\
    Также во время работы над лабораторной, я научился работать с официальной документацией Oracle по встроенной библиотеке Math\cite{oracledocmath}, RandomGenerator\cite{oracledocrandom}, а также ознакомился с базовыми командами *NIX\cite{gnudoc} и Git\cite{gitdoc}.
% -- из biblist
    \newpage
%<<<<<<<<<<<<<<<<<<<<<< КОД РАБОТЫ <<<<<<<<<<<<<<<<<<<<<<<<


%>>>>>>>>>>>>>>>> СПИСОК ЛИТЕРАТУРЫ >>>>>>>>>>>>>>>>>>>>>>>
    \begin{thebibliography}{}
    \bibitem{github} Cсылка на личный репозиторий GitHub: \url{https://github.com/pozitp/itmo-labs/tree/main/prog/lab2}\\
    \bibitem{pokemondb} Ссылка на сайт с информацией о покемонах: \url{https://pokemondb.net}\\
    \bibitem{sedoc} Ссылка на документацию по jar библиотеке с покемонами: \url{https://se.ifmo.ru/~tony/doc/}\\
\end{thebibliography}  % Для соответсвия гост, придется доработать. Нужен файл .bib
%<<<<<<<<<<<<<<<<<<<< СПИСОК ЛИТЕРАТУРЫ <<<<<<<<<<<<<<<<<<<


\end{document}
%<<<<<<<<<<<<<<<< ,,,,,,,,,,,,,,,,,,,,,,, <<<<<<<<<<<<<<<<<
%<<<<<<<<<<<<<<<<<<< СОДЕРЖИМОЕ ОТЧЕТА <<<<<<<<<<<<<<<<<<<<
