\usepackage{wasysym}%%%%%%%%%%%%%%%%%%%%%%%%%%%%%%%%% LAB-5 %%%%%%%%%%%%%%%%%%%%%%%%%%%%%%%%%%
%>>>>>>>>>>>>>>>>>>>>>>>>>> ПЕРЕМЕННЫЕ >>>>>>>>>>>>>>>>>>>>>>>>>>>>>>>>>>>

%>>>>> Информация о кафедре
%\newcommand{\year}{2021 г.}  % Год устанавливается автоматически
\newcommand{\city}{Санкт-Петербург}  %  Футер, нижний колонтитул на титульном листе
\newcommand{\university}{Национальный исследовательский университет ИТМО}  % первая строка
\newcommand{\department}{Факультет программной инженерии и компьютерной техники}  % Вторая строка
\newcommand{\major}{Направление программная инженерия}  % Треьтя строка
% Пусть будет. Проще закоментить лишнее.
\newcommand{\education}{Образовательная программа системное и прикладное программное обеспечение}  % четвертая строка
%\newcommand{\specialization}{}  % пятая строка

%<<<<< Информация о кафедре
%Ашч
%>>>>> Назание работы
\newcommand{\reporttype}{ОТЧЕТ ПО ЛАБОРАТОРНОЙ РАБОТЕ} % тип работы, (главный заголовок титульного листа)
\newcommand{\lab}{Лабораторная работа}          % вид работы
\newcommand{\labnumber}{№ 3--4}                    % порядковый номер работы
\newcommand{\subject}{Программирование}         % учебный предмет
\newcommand{\labtheme}{Принципы объектно-ориентированного программирования SOLID и STUPID}            % Тема лабораторной работы
\newcommand{\variant}{№ 68118}                % номер варианта работы
\newcommand{\student}{Мухамедьяров Артур Альбертович}    % определение ФИО студента
\newcommand{\studygroup}{P3109}                 % определение учебной группы
\newcommand{\teacher}{% принимающий
    Гаврилов А. В.,\\[1mm]% ФИО лектора
    Мустафаева А. В.% ФИО практика
}
%<<<<<<<<<<<<<<<<<<<<<<<<<< ПЕРЕМЕННЫЕ <<<<<<<<<<<<<<<<<<<<<<<<<<<<<<<<<<<


%>>>>>>>>>>>>>>>>>>>>>> ПРЕАМБУЛА >>>>>>>>>>>>>>>>>>>>>>>>>
\include{preamble}
%<<<<<<<<<<<<<<<<<<<<<< ПРЕАМБУЛА <<<<<<<<<<<<<<<<<<<<<<<<<


%%%%%%%%%%%%%%%%%%% СОДЕРЖИМОЕ ОТЧЕТА %%%%%%%%%%%%%%%%%%%%%
%>>>>>>>>>>>>>>> ''''''''''''''''''''''' >>>>>>>>>>>>>>>>>>
\begin{document}


%>>>>>>>>>>>>>>>> ОПРЕДЕЛЕНИЕ НАЗВАНИЙ >>>>>>>>>>>>>>>>>>>>
% Переоформление некоторых стандартных названий
%\renewcommand{\chaptername}{Лабораторная работа}
    \renewcommand{\chaptername}{\lab\ \labnumber} % переименование глав
    \renewcommand{\contentsname}{Содержание} % переименование оглавления
%<<<<<<<<<<<<<<<< ОПРЕДЕЛЕНИЕ НАЗВАНИЙ <<<<<<<<<<<<<<<<<<<<
% \setlength{\itemsep}{0pt} % установка расстояния между строчками в списках можно использовать локально внутри списка списке
% \setlength{\parskip}{0pt} % 
% \setlength{\parsep}{0pt}  % 

%>>>>>>>>>>>>>>>>> ТИТУЛЬНАЯ СТРАНИЦА >>>>>>>>>>>>>>>>>>>>>
    \include{titlepage}
%<<<<<<<<<<<<<<<<< ТИТУЛЬНАЯ СТРАНИЦА <<<<<<<<<<<<<<<<<<<<<


%>>>>>>>>>>>>>>>>>>>>> СОДЕРЖАНИЕ >>>>>>>>>>>>>>>>>>>>>>>>>
% Содержание
    \tableofcontents
%<<<<<<<<<<<<<<<<<<<<< СОДЕРЖАНИЕ <<<<<<<<<<<<<<<<<<<<<<<<<


%%%%%%%%%%%%%%%%%%%%%%% КОД РАБОТЫ %%%%%%%%%%%%%%%%%%%%%%%%
%>>>>>>>>>>>>>>>>>>>'''''''''''''''''>>>>>>>>>>>>>>>>>>>>>
    \newpage
    \Chapter{\lab\ \labnumber}{\labtheme}{}

    \Section{Задание варианта \variant}

%    \begin{center}
%        , , ,
%    \end{center}
%    \noindent
%    \textbf{
%    % Заглавное описание....:
%        Заголовок
%    }
%
%    \textit{
%    % Описание задания...
%        Описание
%    }
    В соответствии с выданным вариантом на основе предложенного текстового отрывка из литературного произведения создать объектную модель реального или воображаемого мира, описываемого данным текстом.
    Должны быть выделены основные персонажи и предметы со свойственным им состоянием и поведением.
    На основе модели написать программу на языке Java.\\
    \textbf{Описание предметной области, по которой должна быть построена объектная модель:}\\
    Незнайка махнул рукой и стал разглядывать красные туфли на ногах собеседника.
    Он заметил, что туфли застегивались на пряжки, которые были сделаны в виде полумесяца со звездой.
    Незнайка принялся рассказывать о том, как мечтал о волшебной палочке, как Кнопочка рассказала ему, что нужно совершать хорошие поступки, и как у него ничего не вышло, потому что он был способен совершать хорошие поступки только ради волшебной палочки, а не бескорыстно.\\
    \textbf{Этапы выполнения работы:}
    \begin{enumerate}
        \item Получить вариант
        \item Нарисовать UML-диаграмму, представляющую классы и интерфейсы объектной модели и их взаимосвязи;
        \item Придумать сценарий, содержащий действия персонажей, аналогичные приведенным в исходном тексте;
        \item \underline{Согласовать диаграмму классов и сценарий с преподавателем;}
        \item Написать программу на языке Java, реализующую разработанные объектную модель и сценарий взаимодействия и изменения состояния объектов.
        При запуске программа должна проигрывать сценарий и выводить в стандартный вывод текст, отражающий изменение состояния объектов, приблизительно напоминающий исходный текст полученного отрывка.
        \item Продемонстрировать выполнение программы на сервере \verb|helios|.
        \item Ответить на контрольные вопросы и выполнить дополнительное задание.
    \end{enumerate}
    Текст, выводящийся в результате выполнения программы \underline{не обязан дословно повторять текст}, полученный в исходном задании.
    Также не обязательно реализовывать грамматическое согласование форм и падежей слов выводимого текста.\\
    Стоит отметить, что \underline{цель разработки} объектной модели \underline{состоит не в выводе текста}, а в эмуляции объектов предметной области, а именно их состояния (поля) и поведения (методы).
    Методы в разработанных классах должны изменять состояние объектов, а \underline{выводимый текст должен являться побочным эффектом}, отражающим эти изменения.\\
    \textbf{Требования к объектной модели, сценарию и программе:}
    \begin{enumerate}
        \item В модели должны быть представлены основные персонажи и предметы, описанные в исходном тексте.
        Они должны иметь необходимые атрибуты и характеристики (состояние) и уметь выполнять свойственные им действия (поведение), а также должны образовывать корректную иерархию наследования классов.
        \item Объектная модель должна реализовывать основные принципе ООП - инкапсуляцию, наследование и полиморфизм.
        Модель должна соответствовать принципам SOLID, быть расширяемой без глобального изменения структуры модели.
        \item Сценарий должен быть вариативным, то есть при изменении начальных характеристик персонажей, предметов или окружающей среды, их действия могут изменяться и отклоняться от базового сценария, приведенного в исходном тексте.
        Кроме того, сценарий должен поддерживать элементы случайности (при генерации персонажей, при задании исходного состояния, при выполнении методов).
        \item Объектная модель должна содержать \underline{как минимум} один корректно использованный элемент \underline{каждого типа} из списка:
        \begin{itemize}
            \item абстрактный класс как минимум с одним абстрактным методом;
            \item интерфейс;
            \item перечисление (enum);
            \item запись (record);
            \item массив или ArrayList для хранения однотипных объектов;
            \item проверяемое исключение.
        \end{itemize}
        \item В созданных классах основных персонажей и предметов должны быть корректно переопределены методы \verb|equals()|, \verb|hashCode()| и \verb|toString()|.
        Для классов-исключений необходимо переопределить метод \verb|getMessage()|.
        \item Созданные в программе классы-исключения должны быть использованы и обработаны.
        Кроме того, должно быть использовано и обработано хотя бы одно unchecked исключение (можно свое, можно из стандартной библиотеки).
        \item При необходимости можно добавить внутренние, локальные и анонимные классы.
    \end{enumerate}
    \begin{figure}[H] % 'H' -- вставить тут же (подключен модуль), обычный вариант: 'htpb'
        \centering
        % { граница для иллюстрации
        % \setlength{\fboxsep}{0pt}% убрать отсутп от границы
        % \setlength{\fboxrule}{1pt}%
        % \fbox{%
        \includegraphics[width=\textwidth]{res/diagram}
        % }} % ограничение области действия параметров
        \caption{UML диаграмма классов}
        \label{fig:enter-label}
    \end{figure}

%    \begin{center}
%        ' ' '
%    \end{center}

    \newpage
    \Section{Выполнение задания.}
    Задание было выполнено в редакторе кода IntejjiJ IDEA, собрано в \verb|jar| файл \verb|lab3.jar| и загружено в Git репозиторий на GitHub.
%    \newpage
    \Subsection{Листинги кода}
    Листинг из файла~\ref{lst:java}
    \lstinputlisting[caption={Исходный код главного класса программы},label={lst:java},language=Java]{../src/Main.java}

%    Листинг в код latex \ref{lst:sql}
%    \begin{lstlisting}[caption={SQL},label={lst:sql}]
%declare @t table(
%  id int
%)
%    \end{lstlisting}

%    Листинг прямо в текст: \lstinline[columns=fixed]{declare}. Либо еще так: \verb|declare|.

%    Вставленное изображение с описанием и шириной по тексту


% Выполнение задания...
    \Section{Результат работы программы.}
%    \Subsection{Первый запуск.}
    \begin{lstlisting}[caption={Результат выполнения программы},label={lst:result}]
    Незнайка махнул рукой
    Незнайка разглядывает красные туфли с пряжкой в виде полумесяц со звездой
    Незнайка рассказывает "о том, как мечтал о волшебной палочке"
    Кнопочка говорит "нужно совершать хорошие поступки бескорыстно"
    но у Незнайка ничего не вышло, так как он делал добро не бескорыстно
    \end{lstlisting}

    \Section{Вывод}
% Вывод...
    Во время выполнения лабораторной работы я изучил основные принципы объектно-ориентированного программирования в Java.
    Освоил работу с наследованием, абстрактными классами, интерфейсами, изучил особенности использования различных модификаторов доступа и методов.
    Рассмотрел принципы SOLID, STUPID, особенности работы с коллекциями, системой обработки исключений и современными возможностями языка, такими как enum и record.
    Полученные знания создают надежную базу для дальнейшего изучения языка программирования и разработки программных решений.
% -- из biblist
    \newpage
%<<<<<<<<<<<<<<<<<<<<<< КОД РАБОТЫ <<<<<<<<<<<<<<<<<<<<<<<<


%>>>>>>>>>>>>>>>> СПИСОК ЛИТЕРАТУРЫ >>>>>>>>>>>>>>>>>>>>>>>
    \begin{thebibliography}{}
    \bibitem{github} Cсылка на личный репозиторий GitHub: \url{https://github.com/pozitp/itmo-labs/tree/main/prog/lab2}\\
    \bibitem{pokemondb} Ссылка на сайт с информацией о покемонах: \url{https://pokemondb.net}\\
    \bibitem{sedoc} Ссылка на документацию по jar библиотеке с покемонами: \url{https://se.ifmo.ru/~tony/doc/}\\
\end{thebibliography}  % Для соответсвия гост, придется доработать. Нужен файл .bib
%<<<<<<<<<<<<<<<<<<<< СПИСОК ЛИТЕРАТУРЫ <<<<<<<<<<<<<<<<<<<


\end{document}
%<<<<<<<<<<<<<<<< ,,,,,,,,,,,,,,,,,,,,,,, <<<<<<<<<<<<<<<<<
%<<<<<<<<<<<<<<<<<<< СОДЕРЖИМОЕ ОТЧЕТА <<<<<<<<<<<<<<<<<<<<
