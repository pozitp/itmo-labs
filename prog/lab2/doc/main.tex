\usepackage{wasysym}%%%%%%%%%%%%%%%%%%%%%%%%%%%%%%%%% LAB-5 %%%%%%%%%%%%%%%%%%%%%%%%%%%%%%%%%%
%>>>>>>>>>>>>>>>>>>>>>>>>>> ПЕРЕМЕННЫЕ >>>>>>>>>>>>>>>>>>>>>>>>>>>>>>>>>>>
%>>>>> Информация о кафедре
%\newcommand{\year}{2021 г.}  % Год устанавливается автоматически
\newcommand{\city}{Санкт-Петербург}  %  Футер, нижний колонтитул на титульном листе
\newcommand{\university}{Национальный исследовательский университет ИТМО}  % первая строка
\newcommand{\department}{Факультет программной инженерии и компьютерной техники}  % Вторая строка
\newcommand{\major}{Направление программная инженерия}  % Треьтя строка
% Пусть будет. Проще закоментить лишнее.
\newcommand{\education}{Образовательная программа системное и прикладное программное обеспечение}  % четвертая строка
%\newcommand{\specialization}{}  % пятая строка

%<<<<< Информация о кафедре

%>>>>> Назание работы
\newcommand{\reporttype}{ОТЧЕТ ПО ЛАБОРАТОРНОЙ РАБОТЕ} % тип работы, (главный заголовок титульного листа)
\newcommand{\lab}{Лабораторная работа}          % вид работы
\newcommand{\labnumber}{№ 2}                    % порядковый номер работы
\newcommand{\subject}{Программирование}         % учебный предмет
\newcommand{\labtheme}{Принципы ООП}            % Тема лабораторной работы
\newcommand{\variant}{№ 68118}                % номер варианта работы
\newcommand{\student}{Мухамедьяров Артур Альбертович}    % определение ФИО студента
\newcommand{\studygroup}{P3109}                 % определение учебной группы
\newcommand{\teacher}{% принимающий
    Гаврилов А. В.,\\[1mm]% ФИО лектора
    Мустафаева А. В.% ФИО практика
}
%<<<<<<<<<<<<<<<<<<<<<<<<<< ПЕРЕМЕННЫЕ <<<<<<<<<<<<<<<<<<<<<<<<<<<<<<<<<<<


%>>>>>>>>>>>>>>>>>>>>>> ПРЕАМБУЛА >>>>>>>>>>>>>>>>>>>>>>>>>
\include{preamble}
%<<<<<<<<<<<<<<<<<<<<<< ПРЕАМБУЛА <<<<<<<<<<<<<<<<<<<<<<<<<


%%%%%%%%%%%%%%%%%%% СОДЕРЖИМОЕ ОТЧЕТА %%%%%%%%%%%%%%%%%%%%%
%>>>>>>>>>>>>>>> ''''''''''''''''''''''' >>>>>>>>>>>>>>>>>>
\begin{document}


%>>>>>>>>>>>>>>>> ОПРЕДЕЛЕНИЕ НАЗВАНИЙ >>>>>>>>>>>>>>>>>>>>
% Переоформление некоторых стандартных названий
%\renewcommand{\chaptername}{Лабораторная работа}
    \renewcommand{\chaptername}{\lab\ \labnumber} % переименование глав
    \renewcommand{\contentsname}{Содержание} % переименование оглавления
%<<<<<<<<<<<<<<<< ОПРЕДЕЛЕНИЕ НАЗВАНИЙ <<<<<<<<<<<<<<<<<<<<
% \setlength{\itemsep}{0pt} % установка расстояния между строчками в списках можно использовать локально внутри списка списке
% \setlength{\parskip}{0pt} % 
% \setlength{\parsep}{0pt}  % 

%>>>>>>>>>>>>>>>>> ТИТУЛЬНАЯ СТРАНИЦА >>>>>>>>>>>>>>>>>>>>>
    \include{titlepage}
%<<<<<<<<<<<<<<<<< ТИТУЛЬНАЯ СТРАНИЦА <<<<<<<<<<<<<<<<<<<<<


%>>>>>>>>>>>>>>>>>>>>> СОДЕРЖАНИЕ >>>>>>>>>>>>>>>>>>>>>>>>>
% Содержание
    \tableofcontents
%<<<<<<<<<<<<<<<<<<<<< СОДЕРЖАНИЕ <<<<<<<<<<<<<<<<<<<<<<<<<


%%%%%%%%%%%%%%%%%%%%%%% КОД РАБОТЫ %%%%%%%%%%%%%%%%%%%%%%%%
%>>>>>>>>>>>>>>>>>>>'''''''''''''''''>>>>>>>>>>>>>>>>>>>>>
    \newpage
    \Chapter{\lab\ \labnumber}{\labtheme}{}

    \Section{Задание варианта \variant}

    \begin{center}
        , , ,
    \end{center}
    \noindent
%    \textbf{
%    % Заглавное описание....:
%        Заголовок
%    }
%
%    \textit{
%    % Описание задания...
%        Описание
%    }
    На основе базового класса \verb|Pokemon| написать свои классы для заданных видов покемонов.
    Каждый вид покемона должен иметь один или два типа и стандартные базовые характеристики:
    \begin{itemize}
        \item очки здоровья (HP)
        \item атака (attack)
        \item защита (defense)
        \item специальная атака (special attack)
        \item специальная защита (special defense)
        \item скорость (speed)
    \end{itemize}

    Классы покемонов должны наследоваться в соответствии с цепочкой эволюции покемонов.
    На основе базовых классов \verb|PhysicalMove|,\verb|SpecialMove| и \verb|StatusMove| реализовать свои классы для заданных видов атак.

    Атака должна иметь стандартные тип, силу (power) и точность (accuracy).
    Должны быть реализованы стандартные эффекты атаки.
    Назначить каждому виду покемонов атаки в соответствии с вариантом.
    Уровень покемона выбирается минимально необходимым для всех реализованных атак.

    Используя класс симуляции боя \verb|Battle|, создать 2 команды покемонов (каждый покемон должен иметь имя) и запустить бой.

    Базовые классы и симулятор сражения находятся в \href{https://se.ifmo.ru/documents/10180/660917/Pokemon.jar/a7ce60af-6ee6-47d0-a95e-e5ed9a697bd2}{jar-архиве}(обновлен 9.10.2018, исправлен баг с добавлением атак и кодировкой).
    Документация в формате javadoc - \href{https://se.ifmo.ru/~tony/doc/}{здесь}.

    Информацию о покемонах, цепочках эволюции и атаках можно найти на сайтах \href{https://poke-universe.ru/}{https://poke-universe.ru}, \href{https://pokemondb.net/}{https://pokemondb.net},\href{https://veekun.com/dex/pokemon}{https://veekun.com/dex/pokemon}

    \subsubsection*{Комментарии}
    Цель работы: на простом примере разобраться с основными концепциями ООП и научиться использовать их в программах.

    Что надо сделать (краткое описание)

    \begin{enumerate}
        \setlength{\itemsep}{0pt} % Сокращение межстрочных расстояний
        \setlength{\parskip}{0pt}
        \setlength{\parsep}{0pt}
        \item Ознакомиться с \href{https://se.ifmo.ru/~tony/doc/}{документацией}, обращая особое внимание на классы \verb|Pokemon| и \verb|Move|.
        При дальнейшем выполнении лабораторной работы читать документацию еще несколько раз.
        \item Скачать файл Pokemon.jar.
        Его необходимо будет использовать как для компиляции, так и для запуска программы.
        Распаковывать его не надо!
        Нужно научиться подключать внешние jar-файлы к своей программе.
        \item Написать минимально работающую программу и посмотреть как она работает.
        \begin{lstlisting}[label={lst:java1}]
            Battle b = new Battle();
            Pokemon p1 = new Pokemon("Чужой", 1);
            Pokemon p2 = new Pokemon("Хищник", 1);
            b.addAlly(p1);
            b.addFoe(p2);
            b.go();
        \end{lstlisting}
        \item Создать один из классов покемонов для своего варианта.
        Класс должен наследоваться от базового класса \verb|Pokemon|.
        В конструкторе нужно будет задать типы покемона и его базовые характеристики.
        После этого попробуйте добавить покемона в сражение.
        \item Создать один из классов атак для своего варианта (лучше всего начать с физической или специальной атаки).
        Класс должен наследоваться от класса \verb|PhysicalMove| или \verb|SpecialMove|.
        В конструкторе нужно будет задать тип атаки, ее силу и точность.
        После этого добавить атаку покемону и проверить ее действие в сражении.
        Не забудьте переопределить метод \verb|describe|, чтобы выводилось нужное сообщение.
        \item Если действие атаки отличается от стандартного, например, покемон не промахивается, либо атакующий покемон также получает повреждение, то в классе атаки нужно дополнительно переопределить соответствующие методы (см.
        документацию).
        При реализации атак, которые меняют статус покемона (наследники \verb|StatusMove|), скорее всего придется разобраться с классом \verb|Effect|.
        Он позволяет на один или несколько ходов изменить состояние покемона или модификатор его базовых характеристик.
        \item Доделать все необходимые атаки и всех покемонов, распределить покемонов по командам, запустить сражение.
    \end{enumerate}


    \begin{center}
        ' ' '
    \end{center}

    \newpage
    \Section{Выполнение задания.}
    Задание было выполнено в редакторе кода, позже собрано с помощью \verb|javac| в \verb|jar| файл \verb|itmo_proga_lab1.jar| непосредственно на сервере.
%    \newpage
    \Subsection{Листинги кода}
    Листинг из файла~\ref{lst:java}
    \lstinputlisting[caption={Исходный код программы},label={lst:java},language=Java]{../src/Main.java}

%    Листинг в код latex \ref{lst:sql}
%    \begin{lstlisting}[caption={SQL},label={lst:sql}]
%declare @t table(
%  id int
%)
%    \end{lstlisting}

%    Листинг прямо в текст: \lstinline[columns=fixed]{declare}. Либо еще так: \verb|declare|.

%    Вставленное изображение с описанием и шириной по тексту.
%    \begin{figure}[H] % 'H' -- вставить тут же (подключен модуль), обычный вариант: 'htpb'
%        \centering
%        % { граница для иллюстрации
%        % \setlength{\fboxsep}{0pt}% убрать отсутп от границы
%        % \setlength{\fboxrule}{1pt}%
%        % \fbox{%
%        \includegraphics[width=\textwidth]{res/UML-class-diagram.png}
%        % }} % ограничение области действия параметров
%        \caption{Caption}
%        \label{fig:enter-label}
%    \end{figure}


% Выполнение задания...
    \Section{Результат работы программы.}
    \Subsection{Первый запуск.}


    \Section{Вывод}
% Вывод...
    Во время выполнения лабораторной работы я изучил синтаксис языка \verb|Java|, встроенную библиотеку \verb|Math|, научислся работать со средством разработки Java (\verb|JDK|). Также в процессе выполения я научился рабоать с типами данных, классами, функциями, массивами и циклами. Полученные мною знания являются необходимой базой для дальнейшего изучения языка и разработки уже более комлпексных проектов.\\
    Также во время работы над лабораторной, я научился работать с официальной документацией Oracle по встроенной библиотеке Math\cite{oracledocmath}, RandomGenerator\cite{oracledocrandom}, а также ознакомился с базовыми командами *NIX\cite{gnudoc} и Git\cite{gitdoc}.
% -- из biblist
    \newpage
%<<<<<<<<<<<<<<<<<<<<<< КОД РАБОТЫ <<<<<<<<<<<<<<<<<<<<<<<<


%>>>>>>>>>>>>>>>> СПИСОК ЛИТЕРАТУРЫ >>>>>>>>>>>>>>>>>>>>>>>
    \begin{thebibliography}{}
    \bibitem{github} Cсылка на личный репозиторий GitHub: \url{https://github.com/pozitp/itmo-labs/tree/main/prog/lab2}\\
    \bibitem{pokemondb} Ссылка на сайт с информацией о покемонах: \url{https://pokemondb.net}\\
    \bibitem{sedoc} Ссылка на документацию по jar библиотеке с покемонами: \url{https://se.ifmo.ru/~tony/doc/}\\
\end{thebibliography}  % Для соответсвия гост, придется доработать. Нужен файл .bib
%<<<<<<<<<<<<<<<<<<<< СПИСОК ЛИТЕРАТУРЫ <<<<<<<<<<<<<<<<<<<


\end{document}
%<<<<<<<<<<<<<<<< ,,,,,,,,,,,,,,,,,,,,,,, <<<<<<<<<<<<<<<<<
%<<<<<<<<<<<<<<<<<<< СОДЕРЖИМОЕ ОТЧЕТА <<<<<<<<<<<<<<<<<<<<
