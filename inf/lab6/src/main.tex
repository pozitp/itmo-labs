% Preamble
\documentclass[letterpaper,9pt]{extarticle}

% Packages
\usepackage{amsmath,amsthm,amssymb}
\usepackage[T1,T2A]{fontenc}
\usepackage[utf8]{inputenc}
\usepackage[english,russian]{babel}
\usepackage{multicol}
\usepackage{geometry}
\usepackage{textcomp}
\usepackage[inline]{enumitem}
\usepackage[table]{xcolor}
\usepackage{bm}
\usepackage{boldline}
\usepackage{wrapfig}
\usepackage{graphicx}

\geometry{top=1cm,bottom=2cm,left=2cm,right=1.5cm, letterpaper}
\linespread{1}\lineskip=0.5em
%\setlength{\columnsep}{0.5cm}
\setlength{\arrayrulewidth}{1pt}
\pagenumbering{gobble}
\renewcommand{\arraystretch}{2}


% Document
\begin{document}
    \begin{center}
        \LARGE{О Т В Е Т Ы , \quadУ К А З А Н И Я , \quadР Е Ш Е Н И Я}
        \rule{\textwidth}{1pt}
    \end{center}
    \vspace{-1.5em}
    \begin{multicols*}{2}[]
        \noindent
        \begin{Large}
            \guillemotleftКВАНТ\guillemotright \ ДЛЯ МЛАДШИХ ШКОЛЬНИКОВ
        \end{Large}\\[1em]
        ЗАДАЧИ\\[0.25em]
        \textit{(см. \guillemotleftКвант\guillemotright\ \textnumero 5)}
        \begin{enumerate}[leftmargin=*,label=\textbf{\arabic*.},wide, labelwidth=!, labelindent=0pt,noitemsep,topsep=0.5em]
            \item Заднее колесо трактора за секунду совершит ровно один оборот.
            \vspace{0.25em}
            \item См.
            таблицу:
%            \vspace{-1em}
            \begin{footnotesize}
                \begin{center}
                    \begin{tabular}{ | m{1.25em} | m{1.25em} | m{1.25em} V{1.5} m{1.25em} | m{1.25em} | m{1.25em} |}
                        \hline
                        & \!\!\textbf{7\!\guillemotleft\!А\!\guillemotright} & \!\!\textbf{7\!\guillemotleft\!Б\!\guillemotright}& \!\!\textbf{7\!\guillemotleft\!В\!\guillemotright}& \!\!\textbf{7\!\guillemotleft\!Г\!\guillemotright}& \bm{$\sum$}\\
                        \hline
                        \!\!\textbf{7\!\guillemotleft\!А\!\guillemotright} & \cellcolor[RGB]{194,194,193}                       & 3:0                                      & 2:1                                      & 2:0                                      & 7:1         \\
                        \hline
                        \!\!\textbf{7\!\guillemotleft\!Б\!\guillemotright} & 0:3                                                & \cellcolor[RGB]{194,194,193}             & 0:0                                      & 2:0                                      & 2:3         \\
                        \hline
                        \!\!\textbf{7\!\guillemotleft\!В\!\guillemotright} & 1:2                                                & 0:0                                                & \cellcolor[RGB]{194,194,193}             & 2:1                                      & 3:3         \\
                        \hline
                        \!\!\textbf{7\!\guillemotleft\!Г\!\guillemotright} & 0:2                                                & 0:2                                                & 1:2                                                & \cellcolor[RGB]{194,194,193}             & 1:6         \\
                        \hline
                    \end{tabular}
                \end{center}
            \end{footnotesize}
            \item Обозначим угол $AOB$ через $\alpha$, тогда $\displaystyle\angle AOD = \frac{\alpha}{2}$, а $\displaystyle\angle DOC = \frac{\pi}{2} - \frac{\alpha}{2}$.
            Угол $COB$ равен $\displaystyle \alpha - \frac{\pi}{2}$, а $\displaystyle\angle COE = \frac{\alpha}{2} - \frac{\pi}{4}$.
            Следовательно, $\displaystyle\angle DOE = \angle DOC + \angle COE = \left( \frac{\pi}{2} - \frac{\alpha}{2} \right) + \left( \frac{\alpha}{2} - \frac{\pi}{4} \right) = \frac{\pi}{4}$.
            \item Разобьем гирьки на 50 соседних гирек.
            Затем эти 50 пар разобьем на две кучки по 25 пар.
            Теперь из первой кучки положим на левую чашку весов более тяжелую гирьку из каждой пары, а на правую --- более легкую.
            Со второй кучкой поступим наоборот --- на левую чашку положим более легкие гирьки из пар, а на правую --- более тяжелые.
            Очевидно, что в результате весы окажутся в равновесии.
            \item Судья не всегда сможет сделать расписание на оставшиеся 2 дня.
            Например, если в первые три дня команды играли
            фывфыв
            так, как показано на рисунке 1 (отрезками соединены номера команд, играющих в этот день), невозможно было бы устроить расписание даже только на четвертый день.
            Действительно, команда с нечетным номером может играть лишь с командами, имеющими нечетные номера, но таких команд три, следовательно, одной из них не с кем будет играть в четвертый\\[1em]
            \textit{(см. \guillemotleftКвант\guillemotright\ \textnumero 6)}
        \end{enumerate}
        \begin{enumerate}[leftmargin=*,label=\textbf{\arabic*.},wide, labelwidth=!, labelindent=0pt,noitemsep]
            \item Смогут.
            Посколько Буратино не хватает $18$ сольдо, а Мальвине не хватает $7$ сольдо, у Мальвины есть по крайней мере $18-7=11$ сольдо.
            Если она добавит их к деньгам
            Пьеро, то денег на букварь, конечно же, хватит.
            \item См.
            рис.
            2, 3.
            \item Число 220 нельзя предстваить в виде суммы двух натуральных чисел, все цифры которых нечетны.
            \item 1665.
            Сумма последних цифр трех исходных трехзначных чисел оканчивается на 5.
            Числа 5 и 25 не представимы в виде суммы трех ненулевых различных цифр.
            Значит, сумма последних цифр тоже равна 15.
            Тогда, сумма средних цифр тоже равна 15, и сумма первых цифр тоже 15.
            Теперь ясно, что после перестановки мы получим три числа, сумма которых 1665.
            фывфыв
            Напоследок предъявим тройку чисел (одну из возможных), которая удовлетворяет условию задачи: 159, 672, 834.
            \item На рисунке 4 король сделал 49 диагональных ходов.
            Доказать, что число 49 максимально возможное, очень просто.
            Каждый диагональный ход проходит через один узел шахматной доски.
            (Узлом мы здесь называем общую точку 4 клеток шахматной доски.)
            Всего узлов 49.
            Два раза пройти через один и тот же узел без самопересечения пути невозможно.
        \end{enumerate}
        МЕТАМОРФОЗЫ ПОСЛЕДОВАТЕЛЬНОСТЕЙ
        \begin{enumerate}[leftmargin=*,label=\textbf{\arabic*.},wide, labelwidth=!, labelindent=0pt,noitemsep]
            \item $\Pi_{9}$, $\Pi_{10}$, $\Pi_{11}$.
            Если считать, что в момент отплытия лодки отправляется автобус \textnumero 0, то турист должен сесть в автобус \textnumero 66.
            (Решение аналогичной задачи см на с. $87-88$ в журнале \guillemotleftКвант\guillemotright\ \textnumero 1/2 за 1993 г.)
            \item $ \displaystyle P_{4}(x) = x^{4} +2x^{3} + x^{2} - \frac{1}{30}$,
            \[
                S_{4}(n) = \frac{1}{5}n^{5}+\frac{1}{2}n^{4}+\frac{1}{3}n^{3}-\frac{1}{30}n=\frac{n(n+1)(2n+1)(3n^{2}+3n-1)}{30}\text{.}
            \]
            \item $\displaystyle S_{k}(0)= \bm{\int}\limits_0^0 P_{k}(x)dx=0$.
            \item \textit{Указание.} Коэффицент при старшей степени многочлена $P_{k}(x)$ равен 1.
            \item $\displaystyle -\frac{1}{2} + \frac{(-1)^{n}}{2\cos\frac{1}{2}}\cos\left( n+ \frac{1}{2} \right) $.\\
            \textit{Указание.}
            \[
                \bm{\sum}_{k=1}^{n}(-1)^{k}\cos k= \frac{1}{2}\bm{\sum}_{k=1}^{n}\big(\cos(\pi+1)k + \cos(\pi-1)k\big)\text{.}
            \]
        \end{enumerate}
        АНАЛОГИИ В ЗАДАЧАХ ПО ФИЗИКЕ\\
        \textbf{1.} $\displaystyle T = \frac{kq\lambda}{R}$ .\hspace{1em}
        \textbf{2.} $T=\sigma R$ .\hspace{2em}
        \textbf{3, 4, 5.} $\Delta W = -W_{0}/2$ .\\[0.5em]
        МОСКОВСКИЙ ФИЗИКО-ТЕХНИЧЕСКИЙ ИНСТИТУТ\\[0.5em]
        \textsf{МАТЕМАТИКА}\\[0.5em]
        \textit{\textsf{Вариант 1}}
        \begin{enumerate}[leftmargin=*,label=\textbf{\arabic*.},wide, labelwidth=!, labelindent=0pt,noitemsep]
            \item $\displaystyle x = \pm \frac{3\pi}{4} + 2\pi k, k \in \bm{Z} $. \textit{Указание.} В каждом из случаев $\cos x > 0$ и $\cos x < 0$ выполните подстановку $\displaystyle t = \frac{\cos 3x}{\cos x}$.
            \item $\displaystyle x < \frac{34-30\sqrt{2}}{23}$, $x \bm{\geq} 3$.
            \item $\displaystyle \frac{5\sqrt{34}}{12}$. \textit{Решение.} Точка $D$ является центром описанной около треугольника $ABC$ окружности, $AB=5\text{, }AC=4$.
            Пусть $O_{1}$ --- центр окружности, вписанный в равнобедренный
        \end{enumerate}
    \end{multicols*}
\end{document}